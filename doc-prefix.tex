% Seleciona o idioma do documento (conforme pacotes do babel)
%\selectlanguage{english}
\selectlanguage{brazil}

% Retira espaço extra obsoleto entre as frases.
\frenchspacing 

% ----------------------------------------------------------
% ELEMENTOS PRÉ-TEXTUAIS
% ----------------------------------------------------------
% \pretextual

% ---
% Capa
% ---
\imprimircapa
% ---

% ---
% Folha de rosto
% (o * indica que haverá a ficha bibliográfica)
% ---
\imprimirfolhaderosto*
% ---

% ---
% Inserir a ficha bibliografica
% ---

% Isto é um exemplo de Ficha Catalográfica, ou ``Dados internacionais de
% catalogação-na-publicação''. Você pode utilizar este modelo como referência. 
% Porém, provavelmente a biblioteca da sua universidade lhe fornecerá um PDF
% com a ficha catalográfica definitiva após a defesa do trabalho. Quando estiver
% com o documento, salve-o como PDF no diretório do seu projeto e substitua todo
% o conteúdo de implementação deste arquivo pelo comando abaixo:
%
% \begin{fichacatalografica}
%     \includepdf{fig_ficha_catalografica.pdf}
% \end{fichacatalografica}

\begin{fichacatalografica}
	\sffamily
	\vspace*{\fill}					% Posição vertical
	\begin{center}					% Minipage Centralizado
	\fbox{\begin{minipage}[c][8cm]{13.5cm}		% Largura
	\small
	\imprimirautor
	%Sobrenome, Nome do autor
	
	\hspace{0.5cm} \imprimirtitulo  / \imprimirautor. --
	\imprimirlocal, \imprimirdata-
	
	\hspace{0.5cm} \thelastpage p. : il. (algumas color.) ; 30 cm.\\
	
	\hspace{0.5cm} \imprimirorientadorRotulo~\imprimirorientador\\
	
	\hspace{0.5cm}
	\parbox[t]{\textwidth}{\imprimirtipotrabalho~--~\imprimirinstituicao,
	\imprimirdata.}\\
	
	\hspace{0.5cm}
		1. Palavra-chave1.
		2. Palavra-chave2.
		2. Palavra-chave3.
		I. Orientador.
		II. Universidade xxx.
		III. Faculdade de xxx.
		IV. Título 			
	\end{minipage}}
	\end{center}
\end{fichacatalografica}
% ---

% ---
% Inserir folha de aprovação
% ---

% Isto é um exemplo de Folha de aprovação, elemento obrigatório da NBR
% 14724/2011 (seção 4.2.1.3). Você pode utilizar este modelo até a aprovação
% do trabalho. Após isso, substitua todo o conteúdo deste arquivo por uma
% imagem da página assinada pela banca com o comando abaixo:
%
% \begin{folhadeaprovacao}
% \includepdf{folhadeaprovacao_final.pdf}
% \end{folhadeaprovacao}
%
\begin{folhadeaprovacao}

  \begin{center}
    {\ABNTEXchapterfont\large\imprimirautor}

    \vspace*{\fill}\vspace*{\fill}
    \begin{center}
      \ABNTEXchapterfont\bfseries\Large\imprimirtitulo
    \end{center}
    \vspace*{\fill}
    
    \hspace{.45\textwidth}
    \begin{minipage}{.5\textwidth}
        \imprimirpreambulo
    \end{minipage}%
    \vspace*{\fill}
   \end{center}
        
   Trabalho aprovado. \imprimirlocal, 1 de novembro de 2021:

   \assinatura{\textbf{\imprimirorientador} \\ Adrian Luna} 
   \assinatura{\textbf{Professor} \\ Ilka Reis}
   \assinatura{\textbf{Professor} \\ Glaura Franco}
      
   \begin{center}
    \vspace*{0.5cm}
    {\large\imprimirlocal}
    \par
    {\large\imprimirdata}
    \vspace*{1cm}
  \end{center}
  
\end{folhadeaprovacao}
% ---

% ---
% Dedicatória
% ---
\begin{dedicatoria}
   \vspace*{\fill}
   \centering
   \noindent
   \textit{Dedicada às pessoas cientistas.}\vspace*{\fill}
\end{dedicatoria}
% ---

% ---
% Agradecimentos
% ---
\begin{agradecimentos}

  Agradeço \\
  ao Adrian, pela orientação; \\
  à Mari, pelas conversas sobre ciência; \\
  aos meus amigos, pelo tempo ocioso; \\
  e à minha família, pela oportunidade.

\end{agradecimentos}
% ---

% ---
% Epígrafe
% ---
\begin{epigrafe}
    \vspace*{\fill}
	\begin{flushright}
		\textit{
      When we can't think for ourselves, we can always quote. \\
      (Ludwig Wittgenstein)
    }
	\end{flushright}
\end{epigrafe}
% ---

% ---
% RESUMOS
% ---

% resumo em português
\setlength{\absparsep}{18pt} % ajusta o espaçamento dos parágrafos do resumo
\begin{resumo}
  A pandemia causada pela COVID-19 motivou o desenvolvimento de novos modelos 
  epidemiológicos, com o objetivo de orientar a tomada de decisão sobre novas 
  políticas públicas no combate à doença.
  Dado que os impactos da pandemia existem não apenas para os indivíduos 
  infectados pela doença, mas também para a sociedade como um todo, 
  muitos desses modelos descrevem consequênciais sócio-estruturais 
  da pandemia, tais quais a quarentena e a superlotação de hospitais.
  Por isso, modelos epidemiológicos compartimentais dividem a população em 
  compartimentos, que descrevem os estados dos indivíduos. Normalmente, os 
  estados são relativos apenas à doença, mas, no caso de modelos para a 
  pandemia, é útil generalizar estados para representar características 
  sócio-estruturais dos indivíduos.
  Este trabalho foi desenvolvido a partir de um desses novos modelos, o 
  SEIQHRF (Susceptible, Exposed, Infected, Quarantined, Hospitalized, 
  Recovered, Fatality). 

  O objetivo foi implementar simulações numéricas do SEIQHRF em grafos, 
  representando redes sociais.
  Além disso, também exemplificamos alguns casos de uso para as simulações.
  Até a data de publicação deste trabalho, só existiam implementações do 
  SEIQHRF considerando populações de misturas homogêneas. 
  Com a nossa implementação, agora é possível fazer simulações do modelo 
  em questão para pequenas populações, considerando estruturas mais complexas 
  de mistura.
  O conteúdo deste trabalho tem a seguinte ordem: 
  uma introdução aos conceitos básicos de modelos epidemiológicos e 
  grafos aleatórios; detalhamentos sobre a implementação do SEIQHRF; 
  ajuste e simulação de um ERGM (exponential random graph model) 
  representando uma escola de ensino médio; exemplos de simulações e 
  testes de hipóteses utilizando o SEIQHRF no grafo em questão.

   \vspace{\onelineskip}
 
   \noindent 
   \textbf{Palavras-chave}: COVID-19. epidemiologia. ERGM. SEIQHRF.
\end{resumo}

% resumo em inglês
\begin{resumo}[Abstract]
 \begin{otherlanguage*}{english}
   This is the english abstract.

   \vspace{\onelineskip}
 
   \noindent 
   \textbf{Keywords}: latex. abntex. text editoration.
 \end{otherlanguage*}
\end{resumo}

% ---
% inserir lista de ilustrações
% ---
\pdfbookmark[0]{\listfigurename}{lof}
\listoffigures*
\cleardoublepage
% ---

% ---
% inserir lista de tabelas
% ---
\pdfbookmark[0]{\listtablename}{lot}
\listoftables*
\cleardoublepage
% ---

%% ---
%% inserir lista de abreviaturas e siglas
%% ---
%\begin{siglas}
%  \item[ABNT] Associação Brasileira de Normas Técnicas
%  \item[abnTeX] ABsurdas Normas para TeX
%\end{siglas}
%% ---

%% ---
%% inserir lista de símbolos
%% ---
%\begin{simbolos}
%  \item[$ \Gamma $] Letra grega Gama
%  \item[$ \Lambda $] Lambda
%  \item[$ \zeta $] Letra grega minúscula zeta
%  \item[$ \in $] Pertence
%\end{simbolos}
%% ---

% ---
% inserir o sumario
% ---
\pdfbookmark[0]{\contentsname}{toc}
\tableofcontents*
\cleardoublepage
% ---

% ----------------------------------------------------------
% ELEMENTOS TEXTUAIS
% ----------------------------------------------------------
\textual
